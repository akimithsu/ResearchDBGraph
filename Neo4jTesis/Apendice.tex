\chapter{Definición de los términos básicos}

\section{Deep Learning}
El Deep Learning permite que los modelos computacionales que se componen de múltiples capas de procesamiento aprendan representaciones de datos con múltiples niveles de abstracción. El Deep Learning descubre una estructura intrincada en grandes conjuntos de datos mediante el uso del algoritmo de retropropagación para indicar cómo una máquina debe cambiar sus parámetros internos que se utilizan para calcular la representación en cada capa a partir de la representación en la capa anterior. Las redes convolucionales profundas han producido avances en el procesamiento de imágenes, video, voz y audio, mientras que las redes recurrentes han arrojado luz sobre datos secuenciales como texto y voz \cite{lecun-author-2015}.
\section{Texto descriptivo}
Los textos descriptivos presentan con claridad y rigor
los rasgos característicos de personas, animales, objetos, lugares, fenómenos o situaciones. Realizar una
buena descripción exige: observar o pensar atentamente sobre lo que se va a describir, seleccionar los
rasgos más característicos de esa realidad (forma, elementos constituyentes, color, tamaño, gusto, olor…),
ordenar los elementos seleccionados (de arriba abajo,
de izquierda a derecha, de delante a atrás…) y redactar la descripción teniendo en cuenta el fin perseguido: objetividad/subjetividad, expresividad \cite{unknown-author-2015B}.
\section{Vector de ruido}
El ruido Gaussiano muestra una densidad de probabilidad que responde a una distribución normal (o distribución de Gauss).
\section{Función de activación}
Es solo una función de cosa que usa para obtener la salida de nodo. También se conoce como función de transferencia .
Se utiliza para determinar la salida de la red neuronal como sí o no. Asigna los valores resultantes entre 0 a 1 o -1 a 1, etc. (dependiendo de la función) \cite{unknown-author-2019}.
%\subsection{Neuronas divergentes}
\section{Datos multimodales}
Los datos multimodales tienen múltiples picos, también denominados modas. Los datos multimodales suelen indicar que aún no se han considerado variables importantes \cite{unknown-author-2018}.
%The \Gls{latex} typesetting markup language is specially suitable 
%for documents that include \gls{maths}.

\section{Inception score}
El Inception Score es una métrica para evaluar automáticamente la calidad de los modelos generativos de imágenes. Se demostró que esta métrica se correlaciona bien con la puntuación humana del realismo de las imágenes generadas a partir del conjunto de datos CIFAR-10. El IS utiliza una red Inception v3 previamente entrenada en ImageNet y calcula una estadística de los resultados de la red cuando se aplica a las imágenes generadas \cite{barratt2018note}.

\section{Frechet Inception Distance (FID)}

El FID \cite{2020arXiv200914075M} calcula la distancia de Frechet entre imágenes sintéticas y del mundo real basándose en las características extraídas de una red Inception v3 previamente entrenada. Un FID más bajo implica una distancia más cercana entre la distribución de la imagen generada y la distribución de la imagen del mundo real \cite{soloveitchik2021conditional}.

 \section{R-precision}
 La R-precision  para evaluar si una imagen generada está bien condicionada a la descripción de texto dada. La precisión R se mide recuperando el texto relevante dada una consulta de imagen.
\chapter{Glosario de Términos}

\textbf{ALI:}Modelo de inferencia aprendida por el adversario.\\
\textbf{Arquitectura :} En el contexto de redes neuronales se refiere al modelamiento de las capas de convolución, datos de entrada, salida y demás partes que componen la red neuronal mencionada.\\
\textbf{CNN:} Red Neuronal Convolucional\\
\textbf{DM-GAN :}Redes Generativas Antagónicas de Memoria Dinámica.\\
\textbf{Framework GAN:} Marcos de aprendizaje automático de una red adversarial generativa.\\
\textbf{GAN :} Red Adversarial Generativa.\\ 
\textbf{LR:} Regresión Logística.\\
\textbf{NB:} Modelo multinomial de Naive Bayes.\\
\textbf{Part Of Speech:} Etiquetamiento de partes del texto descriptivo ingresado.\\
\textbf{Re-ID :} Re-Identification, en deep learning está definido como la asociación de diversas imágenes de una misma persona.\\
\textbf{SVM:} Super Vector Machine.\\
\textbf{Token de Texto:} Al momento de procesar un texto para convertirlo a imágen dejan de ser texto para ingresar como tokens los cuales son un conjunto de características que representan a un texto en específico.\\
\textbf{UTF-8:} Formato estándar de Transformación Unicode usado en la mayoría de preprocesamiento de texto de lenguaje natural.

